%---------------------------------------------------------------------
% This is a sample file for the jotarticle class.
%
% (c) Susanne Cech, Departement Informatik, ETH Zuerich, 
%     susanne.cech@inf.ethz.ch, 2003-02-27
%
%---------------------------------------------------------------------
% last change: 2003-02-27
%---------------------------------------------------------------------

\documentclass{jotarticle}

%---------------------------------------------------------------------
% GENERAL INFORMATION
%---------------------------------------------------------------------
% This template is to be used for JOT-articles only.
%
% Read this documentation carefully, it will simplify your work.
% Some commands are to be used out by the authors, some by the JOT
% team only. I've tagged the commands with [AUTHOR] and [JOT].
%
% To compile your LaTeX-document, I suggest using pdflatex. We will
% use it here as well because it generates directly PDF, thus 
% a clickable document with internal and external links (HTML) is 
% the result.
%
% Please note: When using pdflatex, your picture files need to be
% either in PNG or PDF-format. 
%
% I'm happy about every comment. If you have questions, suggestions,
% comments, etc, please send me an email: susanne.cech@inf.ethz.ch
%---------------------------------------------------------------------


%---------------------------------------------------------------------
% CONFIGURATION
%---------------------------------------------------------------------
% There are two parts configuration parts. The author can enter the 
% first part, that is information regarding the author and the 
% article.
% The second part is to be filled out by the JOT team. Authors please
% leave it as it is, as any non-suitable input will be removed.
%---------------------------------------------------------------------


%---------------------------------------------------------------------
% CONFIGURATION: BY THE AUTHOR
%---------------------------------------------------------------------
% [AUTHOR] Add the title of your article here
%---------------------------------------------------------------------
\title{The jotarticle Class}
%---------------------------------------------------------------------
% [AUTHOR] Add the authors here
%          Several authors can be separated by either \\ or \and
%          Use bold font for the authors name as shown in the example.
%---------------------------------------------------------------------
\author{\textbf{Susanne Cech}, Departement Informatik, 
        ETH Zurich, Switzerland}
%---------------------------------------------------------------------


%---------------------------------------------------------------------
% CONFIGURATION: BY THE JOT TEAM
%---------------------------------------------------------------------
% [JOT] Set first page and last page of the article/column
%---------------------------------------------------------------------
\firstpage{1}
\lastpage{3}
%---------------------------------------------------------------------
% [JOT] Enter the volume and number of the issue. 
%       Here: vol 2, no 3
%---------------------------------------------------------------------
\jotvolume{2}
\jotnumber{2}
%---------------------------------------------------------------------
% [JOT] Enter the month and year when the issues is published.
%       Here: March 2003
%---------------------------------------------------------------------
\jotmonth{3}
\jotyear{2003}
%---------------------------------------------------------------------
% [JOT] [OPTIONAL] Enter how to cite this article 
%       This is generated. In case it does not fit your expectations,
%       you can enter it yourself.
%---------------------------------------------------------------------
\howtocite{\href{http://www.jot.fm/general/JOT_template_LaTeX.tgz}%
{http://www.jot.fm/general/JOT\_template\_LaTeX.tgz}}
%---------------------------------------------------------------------
% [JOT] Set type of paper, i.e. column or article
%       This is needed for the generation of \howtocite
%---------------------------------------------------------------------
\papertype{document}
%---------------------------------------------------------------------
% [JOT] Set the filename. 
%       This is needed for the generation of \howtocite
%---------------------------------------------------------------------
\filename{}
%---------------------------------------------------------------------
% [JOT] Enter the names of the authors as comma-separated list.
%       This is needed for the generation of \howtocite.
%       (I know this is ugly -- I will work on that as soon as type
%       permits it.) 
%---------------------------------------------------------------------
\authornames{Susanne Cech}
%---------------------------------------------------------------------


%---------------------------------------------------------------------
% START OF THE DOCUMENT
%---------------------------------------------------------------------
% This is where the document starts. In this example, I will guide
% you through this sample article with the explanation of possible
% commands you should use.
%---------------------------------------------------------------------


%---------------------------------------------------------------------
\begin{document}
%---------------------------------------------------------------------

%---------------------------------------------------------------------
% ABSTRACT
%---------------------------------------------------------------------
% [AUTHOR] Clearly a job for the author :-)
%          Add your abstract here
%---------------------------------------------------------------------
\begin{abstract}
This little document shows you how to use commands either especially
defined for JOT or just useful. Authors can use this file as template
for their own papers or as reference. In any case, the source file
of this document should also be read.
\end{abstract}
%---------------------------------------------------------------------


%---------------------------------------------------------------------
\section{Link commands}

\subsection{HTML Links}

To include HTML links, you should use this command: \verb|\htmllink{URL}|

\paragraph{Example} \htmllink{http://www.jot.fm}


\subsection{E-Mail Addresses}

To format e-mail addresses, use following command: 
\verb|\maillink{emailaddress}|

\paragraph{Example} \maillink{susanne.cech@inf.ethz.ch}


%---------------------------------------------------------------------
\section{Source Code}

\subsection{Blocks of source code}

To include source code, either the \code{verbatim} or the \code{alltt}
environment can be used. The font is redefined to lightblue color.

\paragraph{Example}
\begin{alltt}
The \textcolor{red}{alltt} is really convenient, since commands are 
still interpreted. For example you can display important things in 
\textit{italics} or \textcolor{red}{color}.
\end{alltt}

\begin{verbatim}
This is the basic verbatim environment. 
Every character is printed as it is.
\end{verbatim}


\subsection{Source code within floating text}

To include source code within floating text, you can use these
two commands:

\begin{verbatim}
\verb|x := x + 1|
\code{x := x + 1}
\end{verbatim}

\paragraph{Examples} \verb|x := x + 1| or \code{x := x + 1}



%---------------------------------------------------------------------
\section{About the bibliography}

The source file shows an example of how a bibliography can look like.
Feel free to use any available form. This is merely shown to see the
order of sections in a document


%---------------------------------------------------------------------
\section{About the biography}


Keep the bio short. Include a picture if you wish (size 100x100 pixels,
solution 72 dpi); if so, make sure it is of good enough quality at 
that size and include it in the document. Include email for at 
least one author, set it up as HTML link. Use this command: 

\begin{verbatim}
\abouttheauthor[picture]{name of the author}{text about the author}
\end{verbatim}

\paragraph{Example} See in the section ``About the authors''.


%---------------------------------------------------------------------
% BIBLIOGRAPHY
%---------------------------------------------------------------------
\begin{thebibliography}{XXXXX}

\bibitem[]{}

\end{thebibliography}


%---------------------------------------------------------------------
% ABOUT THE AUTHORS
%---------------------------------------------------------------------

% This command starts the ``About the Authors'' section.
\abouttheauthors

\abouttheauthor[susi]{Susanne Cech}{is a PhD student and research
assistant at the Chair of Software Engineering at the ETH Zurich,
Switzerland. She can be reached at
\maillink{susanne.cech@inf.ethz.ch}. 
See also \htmllink{http://se.inf.ethz.ch}.}



%---------------------------------------------------------------------
\end{document}
%---------------------------------------------------------------------
